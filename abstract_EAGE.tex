\documentclass{vie16}
\usepackage{wrapfig}
\usepackage{float}
\usepackage{amsmath}
\usepackage[table,xcdraw]{xcolor}
\usepackage{booktabs}
\usepackage{multirow}
\usepackage{adjustbox}

% Path to figures
\graphicspath{{Fig/}}

% COMENTAR ANTES DE SUBMETER!
% Comando temporarios
\usepackage{color}
\usepackage[normalem]{ulem}
\newcommand{\new}[1]{\textcolor{blue}{#1}}
\newcommand{\old}[1]{\textcolor{MyDarkViolet}{\sout{#1}}}
\newcommand{\att}[1]{\textcolor{red}{#1}}
\newcommand{\js}[1]{\textcolor{darkgreen}{#1}}
\newcommand{\obs}[1]{\textcolor{orange}{#1}}
\definecolor{MyDarkViolet}{rgb}{0.602,0,0.602}
\definecolor{orange}{RGB}{255,127,0}
\definecolor{darkgreen}{RGB}{0,127,0}

% For pdf hyperlinks
\usepackage{hyperref}
\hypersetup{colorlinks=true}
\hypersetup{citecolor=blue}
\hypersetup{pdftitle={Development of lock-in resistivimeter
for measuring resistivity and induced polarization under
high electrical noise}}
\hypersetup{pdfauthor={J.~M.~Oliveira, W.~C.~Ferreira and F.~Hiodo}}
% For making a print without color links
\hypersetup{linkcolor=black,citecolor=black,filecolor=black,urlcolor=black}
\hypersetup{linkcolor=blue,citecolor=blue,filecolor=blue,urlcolor=blue}
\usepackage{url}
\begin{document}

\title{Development of lock-in resistivimeter for measuring
resistivity and induced polarization under high electrical noise}
\author{J.~M.~Oliveira, W.~C.~Ferreira and F.~Hiodo}
\maketitle

% Nota: geralmente a EAGE limita o numero de caracteres no abstract. 

\begin{abstract}

	The Cole-Cole RC circuit model and the reduced clay model were
	tested with three levels of white noise in order to develop a low
	cost lock-in resistivimeter. The device has maximum power of 60
	watts and showed reliable stability under $67 \%$ and $11 \%$ of
	white noise in the respective models. The equipment was made with
	low cost components and yet capable of providing a frequency sweep
	from $13$ to $120$Hz to the transmitter which allows better signal
	sampling. This frequency bandwidth was used to find the less time
	consuming operational settings able to recover the Cole-Cole
	exponent and relaxation time, in other words, which are the minimum
	number of frequency samples that is necessary to reconstruct the
	Cole-Cole curve. An inversion scheme based on the partial sweep of
	the solution space was implemented. We found that $16$
	frequency samples evenly spaced are enough to recover the relaxation
	time with variations below one order of magnitude. Therefore, the
	results showed good performance for the developed lock-in
	resistivimeter. The suggested applications to such equipment is not
	only restricted to field measurements but also to lab samples
	studies related to Spectral Induced Polarization (SIP), resistivity
	and Induced Polarization (IP) under high electrical noise.  
\end{abstract}	
		

\newpage


\section{Introduction}
Urban regions are characterized by a high noise content to the
geoelectrical methods mainly due to presence of power supply,
grounding and stray voltages. It's also known that intrusive methods
are affected by pavimentation and buildings which impose restrains in
the superficial area making them more susceptible to noise. The
objectives of this work is the design of a synchronous detection
resistivimeter from low cost components to conduct resistivity and IP
surveys using low power supply.  As observed in Figure 1 the equipment
basically consists of a transmitter and a receiver with two channels
each. The reference channel is fed directly in the receiver while the
signal channel is injected from the transmitter into the subsurface
and then to the receiver. An ideal lock-in transmitter would have an
infinite frequency sweep without any distortion and would deliver
constant power for all frequencies. However, due to physical
limitation our transmitter is capable to provide a stable power supply
close to 60W for the frequency bandwidth of 13 to 120Hz with minimal
distortion.  Bellow the 13Hz threshold the electrical sine wave
distortion increases drastically and above the 120Hz there is little
gain of information due to logarithmic nature of frequency.

\section{Sincronous Detection}
The synchronous detection also known as \textit{lock-in} is sensitive
to the difference of frequency and phase between the two inputs
as showed by \citet{meade13}. The voltage reach it's peak when the
correlation in the frequency domain is maximum therefore it can
track signals with the same characteristics of the reference channel
even under intense noise since the noise doesn't have a correlated
phase with the reference. Inside the receiver, the reference channel is
convert to a unitary square wave then is multiplied by the demodulated
signal as showed in the complete scheme in Figure \ref{f:Lock-in}. The
resulting DC voltage on the display is the output of the low-pass
filter of cut off frequency no greater than $\omega_{R}$ 
 as the equation \ref{e:H(w)} describes. % tá esquisito esse paragrafo, mutas ideais misturadas... mas não to conseguindo junta as ideias pra corrigir ele

\begin{equation}
	v_{p(t)} = V_{S}V_{R}\lbrace
								\cos \left[ 
											\left( \omega_{R} + \omega_{S} \right)t + (\phi_{R} + \phi_{S})
									 \right]
									+ 
								\cos \left[ 
											\left( \omega_{R} - \omega_{S} \right)t + (\phi_{R} - \phi_{S})
									 \right]
					     \rbrace
	\label{e:H(w)}
\end{equation}

where $V_{S}$, $\phi_{S}$ and $\omega_{S}$ are the respectively amplitude, phase and angular frequency 
of the measured signal wave and $V_{R}$, $\phi_{R}$ and $\omega_{R}$ are the respectively amplitude, 
phase and angular frequency of the reference wave.


\begin{figure}[H]
	\centering
	\includegraphics[keepaspectratio=true,scale=0.3]{Sistema-Inside_Lock_In}
	\caption{Scheme of synchronous detection. The arrows show the way followed by the source signal from the transmitter injected into the receiver reference channel (\textit{Up}) and signal channel (\textit{down}). In the reference channel $1^{,}$ is a optional amplifier or attenuator. In the reference channel $1$ is a phase-shifter for correcting the added phase from the detector, $2$ output a square wave from the sine wave input, $3$ is the multiplier and  $4$ the low-pass filter led to the display. }
	\label{f:Lock-in}
\end{figure}


Therefore, analizing the equation \ref{e:H(w)} we note two
special cases; (1) the output will be proportional only to the
resistivity if the two inputs are in phase; (2) the output will be
proportional only to the imaginary resistivity if they are in
quadrature which is related to polarization effects.  % Achei uma boa referencia nos papers do TG
%imaginary resistivity pode ser electric permittivity, NÃO REMOVA ESSE PERCENT!


\section{Inversion}
As mentioned before, due to practical limitations our resistivimeter frequencies 
range from $13$ to $120$Hz therefore an inversion scheme of the Cole-Cole parameters 
was used to verify how many frequency samples in this range were needed to
re-build the Cole-Cole curve. The implemented technique was partial sweep scheme,
it consists of choosing specific values for the relaxation time
($\tau$) and for the Cole-Cole exponent ($c$) inside a prior
established interval and then computing a \textit{l2-norm} between
each modelled curve and the data. In the last step, it's built an map
of the solution space which is analized to look for the minima
positions.  Using equation \ref{e:CC} we define $c \in [0:1]$ and
$\tau \in [10^{-4}:10^{4}]$ based on physical meaningful values for each
parameter \att{REFERENCIA}
. It is important to highlight that we could have more than
one relaxation time in the data, however to simplify our analysis only
one relaxation time (with possible values inside the prior defined
range) is used for each Cole-Cole curve.


%    Smith, M.J., 1980, Comparison of induced polarization measurements over the Elura orebody, The Geophysics of the Elura Orebody, Cobar NSW, ASEG, 1980, 77-80.

\begin{equation}
	\sigma_{e} = \sigma_{0}
			\left[
			1 + \frac{m}{1-m}
						\left( 
								1 - \frac{1}{1+ \left( j\omega\tau \right)^{c} }	
						\right) 			
			\right]
	\label{e:CC}
\end{equation}

where $\sigma_{e}$ is the complex conductivity, $\sigma_{0}$ is the
conductivity at zero frequency, $m$ is the chargeability, $\omega$ is 
the angular frequency, $\tau$ is the relaxation time and $c$ is the 
Cole-Cole exponent.


\section{Results}


% transfer function do equipamento
% synchronous detection baixa deriva termica, experimento
% durou horas

% Eu até posso calcular. Dependendo de quantas folhas sobrarem.
%Chargeability could be found from the measured data by the equation 
%\begin{equation}
%	F_{E} = \frac{ |V_{(\omega_{1})} - V_{(\omega_{2})}| }{V_{(\omega_{1})}}
%	\label{e:FE}
%\end{equation}


To verify it's performance in noisy areas was conducted an experiment with scaled models where were
tested how much it's measurements are affected by white noise. The results of table \ref{t:scaled_models} show
a good performance even under $67\%$ in the RC circuit of $12 \mu$F. As result of such high capacitance, the in-phase 
output was working as the quadrature and vise versa. The outcome of a model more close to data expected from field survey, 
a clay tank of $1125$ cm$^3$, also points to application even under $11\%$ of white noise as showed in table \ref{t:scaled_models}.

\begin{table}[H]
\centering
\caption{Measured voltage on scaled models with three levels of white noise.}
\label{t:scaled_models}
\begin{tabular}{cccl|ccc}
\multirow{2}{*}{\textbf{Noise level (\%)}}  & \multicolumn{2}{c}{\textbf{Output}}     &  & \textbf{Noise Level (\%)} & \multicolumn{2}{c}{\textbf{Output}}     \\
                         & \textbf{In Phase}  & \textbf{Quadrature} &  &  	 				& \textbf{In Phase} & \textbf{Quadrature} 	\\
\multirow{4}{*}{0}	 		& 1.52              & 8.45                &  & 	\multirow{4}{*}{0}	&  5.24				&		1.80			\\
					 		& 1.60              & 8.50                &  &						&  4.75				&		1.90			\\
					 		& 1.44              & 8.46                &  &						&  4.80				&		1.92			\\
					 		& 1.48              & 8.40                &  &						&  5.30		    	&		1.81	 
                                         \\ \hline                                   
\multirow{4}{*}{40}	 		& 1.06              & 8.42                &  & 	\multirow{4}{*}{11}	&	4.70			&		1.32			\\
					 		& 1.28              & 8.41                &  &						&	4.60			&		1.20			\\
					 		& 1.29              & 8.37                &  &						&	4.68			&		1.26			\\
					 		& 1.30              & 8.45                &  &						&	4.50			&		1.28			  
                                         \\ \hline
\multirow{4}{*}{67}	 		& 1.29              & 8.48                &  & 	\multirow{4}{*}{27}	&	2.30			&		1.30			\\
					 		& 1.28              & 8.48                &  &						&	2.05			&		1.20			\\
					 		& 1.29              & 8.48                &  &						&	2.12			&		1.24			\\
					 		& 1.30              & 8.50                &  &						&	2.50			&		1.32			  
                                          \\ \hline
\end{tabular}
\end{table}

As for the Cole-Cole paramenters, the effect of noise and of sample interval were tested for three models using four values of Gaussian noise and for four different frequency sampling at $5\%$ of Gaussian noise respectively. These three models do not felt inside the frequency band of the resistivimeter as result the characteristic Cole-Cole curve can't be observed instead the measured data approach an straight line. Surprisingly the exponent of Cole-Cole, whose indicate a dependence of complex conductivity with frequency \att{REFERENCIA} was very well determinated in all cases, 
even under $250\%$ of noise.

In other hand the relaxation time is strongly affected by noise. In table \ref{t:INV-noise_effect} we can see that it's precision vary with the 
percent of noise while it's accuracy stay roughly the same with the exception of the most extreme case of $250\%$ of noise. Nonetheless the best 
results of all models recover both parameters within $10\%$ of uncertainty supporting it's application for SIP.

\begin{table}[H]
\centering
\caption{Noise effect on the inversion of Cole-Cole parameters}
\label{t:INV-noise_effect}
\begin{tabular}{@{}|c|c|c|c|c|c|c|@{}}
\multicolumn{3}{|c|}{\textbf{Sinthetic Model}}                             & \multirow{2}{*}{\textbf{\begin{tabular}[c]{@{}c@{}}Noise\\ (\%)\end{tabular}}} & \multicolumn{3}{c|}{\textbf{Calculated Model}}                            \\
\textbf{Model}     & \textbf{$\tau$}          & \textbf{$e_{cc}$}          &                                                                                & \textbf{$\tau$} & \textbf{$e_{cc}$} & \textbf{$\tau$ deviation}           \\  \hline
\multirow{4}{*}{A} & \multirow{4}{*}{567.243} & \multirow{4}{*}{0.0452261} & 5                                                                              & 567.243         & 0.0452261         & 567 \textless$\tau$\textless 860    \\
                   &                          &                            & 10                                                                             & 517.092         & 0.0452(5)         & 113 \textless$\tau$\textless 3405   \\
                   &                          &                            & 25                                                                             & 517.092         & 0.045(1)          & 56 \textless$\tau$\textless 5670    \\
                   &                          &                            & 250                                                                            & 517.092         & 0.045(2)          & 170 \textless$\tau$\textless 11344  \\ \hline
\multirow{4}{*}{B} & \multirow{4}{*}{2.89942} & \multirow{4}{*}{0.0452261} & 5                                                                              & 2.89942         & 0.0452(5)         & 1.15 \textless$\tau$\textless 11.6  \\
                   &                          &                            & 10                                                                             & 2.89942         & 0.045(1)          & 2.3 \textless$\tau$\textless 4.4    \\
                   &                          &                            & 25                                                                             & 2.89942         & 0.045(5)          & 1.8 \textless$\tau$\textless 5.8    \\
                   &                          &                            & 250                                                                            & 2.89942         & 0.04(1)           & 1.8 \textless$\tau$\textless 14500  \\ \hline
\multirow{4}{*}{C} & \multirow{4}{*}{622.257} & \multirow{4}{*}{0.834171}  & 5                                                                              & 622.257         & 0.834171          & 622.257                             \\
                   &                          &                            & 10                                                                             & 622.257         & 0.834171          & 622.257                             \\
                   &                          &                            & 25                                                                             &                 &                   &                                     \\
                   &                          &                            & 250                                                                            & 622.257         & 0.8(1)            & 435 \textless $\tau $\textless 1245
\end{tabular}
\end{table}

The number of samples needed in a survey also were studied with $5\%$ of noise, considered a bad %adverse?
field situation \att{REFERENCIA}. As showed in table \ref{t:INV-sample_effect} with $16$ samples evenly distributed the relaxation 
time was within the same order of magnitude of the synthetic model. Despite being less precise, $8$ frequency samples also had similar uncertainty therefore being an viable option as well.

\begin{table}[H]
\centering
\caption{Sampling effect on the inversion of Cole-Cole parameters}
\label{t:INV-sample_effect}
\begin{tabular}{@{}|c|c|c|c|c|c|c|c|@{}}
\multicolumn{3}{|c|}{\textbf{Sinthetic Model}}                             & \multirow{2}{*}{\textbf{Sampling}} & \multicolumn{3}{c|}{\textbf{Calculated Model}}                          & \multirow{2}{*}{\textbf{Missfit}}    \\
\textbf{Model}     & \textbf{$\tau$}          & \textbf{$e_{cc}$}          &                                    & \textbf{$\tau$} & \textbf{$e_{cc}$} & \textbf{$\tau$ deviation}         &                                      \\ \hline
\multirow{4}{*}{A} & \multirow{4}{*}{567.243} & \multirow{4}{*}{0.0452261} & 2                                  & 517.092         & 0.045(1)          & 113 \textless$\tau$\textless 1702 & 0.02                                 \\
                   &                          &                            & 4                                  & 474.313         & 0.045(1)          & 454 \textless$\tau$\textless 1702 & 0.01                                 \\
                   &                          &                            & 8                                  & 567.243         & 0.0452261         & 568 \textless$\tau$\textless 850  & 0.02                                 \\
                   &                          &                            & 16                                 & 517.092         & 0.045(1)          & 453 \textless$\tau$\textless 568  & 0.03                                 \\ \hline
\multirow{4}{*}{B} & \multirow{4}{*}{2.89942} & \multirow{4}{*}{0.0452261} & 2                                  & 2.89942         & 0.045(1)          & 0.5 \textless$\tau$\textless 2.9  & 0.007                                \\
                   &                          &                            & 4                                  & 2.89942         & 0.045(1)          & 0.5 \textless$\tau$\textless 2.9  & 0.013                                \\
                   &                          &                            & 8                                  & 2.89942         & 0.045(1)          & 1.5 \textless$\tau$\textless 11.6 & 0.015                                \\
                   &                          &                            & 16                                 & 2.89942         & 0.0452261         & 2.3 \textless$\tau$\textless 2.9  & \textless $10^{-5}$                  \\ \hline
\multirow{4}{*}{C} & \multirow{4}{*}{622.257} & \multirow{4}{*}{0.834171}  & 2                                  & 622.257         & 0.834171          & 622.257                           & \multirow{4}{*}{\textless $10^{-5}$} \\
                   &                          &                            & 4                                  & 622.257         & 0.834171          & 622.257                           &                                      \\
                   &                          &                            & 8                                  & 622.257         & 0.834171          & 622.257                           &                                      \\
                   &                          &                            & 16                                 & 622.257         & 0.834171          & 622.257                           &                                   \\  \hline
\end{tabular}
\end{table}

For the inversion of Cole-Cole parameters the percent of noise in the synthetic model was considered as being the output from ours resistivimeter. Thus filtering capabilities of our resistivimeter wasn't taken account. By the noise rejection displayed in table \ref{t:scaled_models} the viability of SIP is reinforced since a high level of noise in the output consequently implying in higher level of noise in the input. 

\section{Conclusions}

Compared to others instruments in use today, the low power
resistivimeter was able to reject great amount of noise in scaled
models. The results show stable measurements up to $67\%$ of white
noise in RC circuit and $11 \%$ in the clay experiment. These results
strongly support the viabilty of such low cost device in geoelectrical
surveys even in high noise environments. The inversion study for the
Cole-Cole parameters points towards the viability of the equipment in
others applications such as the multifrequency resistivimeter for
Spectral Induced Polarization with a bandwith of $13$ to $120$Hz. We
showed that $8$ frequency samples evenly distributed were enough to
recover the Cole-Cole parameters.



%A  intensa rejei\c{c}\~ao de ru\'idos pelo m\'etodo de detec\c{c}\~ao s\'incrona permite a manufatura de eletrorresistiv\'imetros com baixo custo de mercado e baixa pot\^encia de opera\c{c}\~ao se comparado a outros equipamentos em uso no mercado. 
		
%		Este tipo de eletrorresistiv\'imetro apresentou estabilidade para at\'e $11\%$ de ru\'ido branco para as medidas da argila no modelo em escala reduzida. Este nivel de ru\'ido representa uma situa\c{c}\~ao muito desfavor\'avel para a aquisi\c{c}\~ao geoel\'etrica, mostrando que o desempenho do equipamento \'e suficientemente confi\'avel para sua aplica\c{c}\~ao em ambientes ruidosos.
		
%		Dada a possibilidade de aquisi\c{c}\~ao com v\'arias frequ\^encias, foi analisado o potencial da aplica\c{c}\~ao do eletrorresisitiv\'imetro na medida de polariza\c{c}\~ao ind\'uzida espectral. Os resultados obtidos pela invers\~ao dos par\^ametro de Cole-Cole, sugerem que a faixa de $13$ a $110$ Hertz \'e suficiente para a recupera\c{c}\~ao dos par\^ametros do modelo.
		
%		De maneira a aperfei\c{c}oar o equipamento para uso em campo, \'e sugerido a digitaliza\c{c}\~ao da sa\'ida, desta forma a fase poderia ser medida diretamente no equipamento eliminando a etapa de calibra\c{c}\~ao. Esta medida tamb\'em que os dados sejam salvos no equipamento, acelerando o seu processamento. 

\bibliography{abstract_EAGE}

\section{Acknowledgements (Optional)}

This is the first sentence of the acknowledgements.

% \begin{thebibliography}{6pt}
%   \bibitem[{<reference>}]{<cite>} ...
% \end{thebibliography}
%
% or
%
% \bibliography{...}

\end{document} 